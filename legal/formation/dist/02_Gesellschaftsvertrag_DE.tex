\hypertarget{gesellschaftsvertrag-operating-agreement}{%
\section{GESELLSCHAFTSVERTRAG (Operating
Agreement)}\label{gesellschaftsvertrag-operating-agreement}}

\hypertarget{der-blauweiss-edv-llc}{%
\subsection{der BLAUWEISS-EDV LLC}\label{der-blauweiss-edv-llc}}

\hypertarget{eine-limited-liability-company-nach-dem-recht-des-staates-texas}{%
\subsubsection{Eine Limited Liability Company nach dem Recht des Staates
Texas}\label{eine-limited-liability-company-nach-dem-recht-des-staates-texas}}

\emph{{[}Übersetzung zu Dokumentationszwecken -- das englische Original
ist rechtlich maßgeblich{]}}

\begin{center}\rule{0.5\linewidth}{0.5pt}\end{center}

Dieser Gesellschaftsvertrag (das ``\textbf{Agreement}'') wird zum
\_\_\_\_\_\_\_\_\_\_\_\_\_\_\_\_, 2025 (das
``\textbf{Wirksamkeitsdatum}'') zwischen den hierin bezeichneten
Gesellschaftern geschlossen.

\textbf{PRÄAMBEL:} Die Blauweiss-EDV LLC (die ``\textbf{Gesellschaft}'')
wurde am 6. November 2025 als Limited Liability Company nach dem Recht
des Staates Texas durch Einreichung einer Gründungsurkunde beim Texas
Secretary of State gegründet (Aktenzeichen 806293404).

Die Gesellschafter beabsichtigen, die Bedingungen für den Betrieb der
Gesellschaft sowie die Rechte und Pflichten der Gesellschafter
festzulegen.

\begin{center}\rule{0.5\linewidth}{0.5pt}\end{center}

\hypertarget{artikel-i-organisation}{%
\subsection{ARTIKEL I -- ORGANISATION}\label{artikel-i-organisation}}

\hypertarget{firma}{%
\subsubsection{§ 1.1 -- Firma}\label{firma}}

Die Firma der Gesellschaft lautet Blauweiss-EDV LLC.

\hypertarget{hauptsitz}{%
\subsubsection{§ 1.2 -- Hauptsitz}\label{hauptsitz}}

Der Hauptsitz der Gesellschaft befindet sich an einem von den
Gesellschaftern jeweils zu bestimmenden Ort. Das eingetragene Büro
entspricht den Angaben in der Gründungsurkunde.

\hypertarget{unternehmensgegenstand}{%
\subsubsection{§ 1.3 --
Unternehmensgegenstand}\label{unternehmensgegenstand}}

Gegenstand der Gesellschaft ist die Erbringung von IT-Dienstleistungen,
einschließlich aber nicht beschränkt auf Softwareentwicklung,
Cloud-Architektur, DevOps-Beratung, Lösungen im Bereich künstlicher
Intelligenz sowie alle anderen von den Gesellschaftern bestimmten
rechtmäßigen Geschäftstätigkeiten.

\hypertarget{dauer}{%
\subsubsection{§ 1.4 -- Dauer}\label{dauer}}

Die Gesellschaft besteht auf unbestimmte Zeit, sofern sie nicht gemäß
diesem Vertrag oder anwendbarem Recht aufgelöst wird.

\begin{center}\rule{0.5\linewidth}{0.5pt}\end{center}

\hypertarget{artikel-ii-gesellschafter-und-gesellschaftsanteile}{%
\subsection{ARTIKEL II -- GESELLSCHAFTER UND
GESELLSCHAFTSANTEILE}\label{artikel-ii-gesellschafter-und-gesellschaftsanteile}}

\hypertarget{gruxfcndungsgesellschafter}{%
\subsubsection{§ 2.1 --
Gründungsgesellschafter}\label{gruxfcndungsgesellschafter}}

Die Gründungsgesellschafter der Gesellschaft und ihre jeweiligen
Gesellschaftsanteile sind wie folgt:

\begin{quote}
\textbf{Michael Clement Matejka} -- 50\% Gesellschaftsanteil\\
106 Stratford St, Houston, TX 77006, USA\\
US-Staatsbürger
\end{quote}

\begin{quote}
\textbf{Ian Bennett Matejka} -- 50\% Gesellschaftsanteil\\
Houston, Texas, USA\\
US-Staatsbürger
\end{quote}

\hypertarget{aufnahme-weiterer-gesellschafter}{%
\subsubsection{§ 2.2 -- Aufnahme weiterer
Gesellschafter}\label{aufnahme-weiterer-gesellschafter}}

Weitere Gesellschafter können mit einstimmiger schriftlicher Zustimmung
aller bestehenden Gesellschafter und durch Unterzeichnung einer Änderung
dieses Vertrags aufgenommen werden, die die Aufnahme des neuen
Gesellschafters und die Anpassung der Gesellschaftsanteile
widerspiegelt.

\begin{center}\rule{0.5\linewidth}{0.5pt}\end{center}

\hypertarget{artikel-iii-beteiligungsoption}{%
\subsection{ARTIKEL III --
BETEILIGUNGSOPTION}\label{artikel-iii-beteiligungsoption}}

\hypertarget{einruxe4umung-der-option}{%
\subsubsection{§ 3.1 -- Einräumung der
Option}\label{einruxe4umung-der-option}}

Die Gesellschafter räumen hiermit \textbf{Wolfram Laube}
(``\textbf{Optionsberechtigter}''), einem österreichischen Staatsbürger
wohnhaft in Unterer Stadtplatz 18, 4780 Schärding, Österreich, eine
Option (die ``\textbf{Beteiligungsoption}'') zum Erwerb eines
Gesellschaftsanteils an der Gesellschaft ein, vorbehaltlich der in
diesem Artikel III festgelegten Bedingungen.

\hypertarget{ausuxfcbungsbedingungen}{%
\subsubsection{§ 3.2 --
Ausübungsbedingungen}\label{ausuxfcbungsbedingungen}}

Die Beteiligungsoption kann nur bei Erfüllung der folgenden Bedingung
ausgeübt werden: Der Abschluss des derzeit gegen den Optionsberechtigten
beim Landesgericht Ried im Innkreis, Österreich, anhängigen
Insolvenzverfahrens (Konkursverfahren GZ 17 S 35/25 s) durch
rechtskräftigen Gerichtsbeschluss, der die Beendigung oder Aufhebung
dieses Verfahrens bestätigt.

\hypertarget{ausuxfcbungsverfahren}{%
\subsubsection{§ 3.3 --
Ausübungsverfahren}\label{ausuxfcbungsverfahren}}

Nach Erfüllung der Ausübungsbedingung kann der Optionsberechtigte die
Beteiligungsoption durch schriftliche Mitteilung an alle Gesellschafter
ausüben, begleitet von Dokumenten, die den Abschluss des
Insolvenzverfahrens belegen. Bei gültiger Ausübung wird der
Optionsberechtigte als Gesellschafter aufgenommen, und die
Gesellschaftsanteile werden so angepasst, dass jeder der drei
Gesellschafter einen Anteil von 33,33\% hält (oder eine andere von allen
Parteien einstimmig vereinbarte Aufteilung).

\hypertarget{ausuxfcbungspreis}{%
\subsubsection{§ 3.4 -- Ausübungspreis}\label{ausuxfcbungspreis}}

Der Ausübungspreis für die Beteiligungsoption beträgt einen US-Dollar
(\$1,00) als Gegenleistung für die Beiträge des Optionsberechtigten zur
Gesellschaft als Independent Contractor und in Anerkennung seiner Rolle
beim Aufbau der Geschäftsbeziehungen und technischen Fähigkeiten der
Gesellschaft.

\hypertarget{optionsfrist}{%
\subsubsection{§ 3.5 -- Optionsfrist}\label{optionsfrist}}

Die Beteiligungsoption bleibt für einen Zeitraum von zwei (2) Jahren
nach Erfüllung der Ausübungsbedingung ausübbar. Wird sie innerhalb
dieses Zeitraums nicht ausgeübt, verfällt die Beteiligungsoption.

\hypertarget{contractor-verhuxe4ltnis}{%
\subsubsection{§ 3.6 --
Contractor-Verhältnis}\label{contractor-verhuxe4ltnis}}

Bis zur Ausübung der Beteiligungsoption erbringt der Optionsberechtigte
Leistungen für die Gesellschaft als Independent Contractor gemäß einem
separaten Independent Contractor Agreement. Dieses Vertragsverhältnis
begründet keine Gesellschafterrechte oder -pflichten vor gültiger
Ausübung der Beteiligungsoption.

\begin{center}\rule{0.5\linewidth}{0.5pt}\end{center}

\hypertarget{artikel-iv-geschuxe4ftsfuxfchrung}{%
\subsection{ARTIKEL IV --
GESCHÄFTSFÜHRUNG}\label{artikel-iv-geschuxe4ftsfuxfchrung}}

\hypertarget{gesellschaftergefuxfchrte-gesellschaft}{%
\subsubsection{§ 4.1 -- Gesellschaftergeführte
Gesellschaft}\label{gesellschaftergefuxfchrte-gesellschaft}}

Die Gesellschaft wird von ihren Gesellschaftern geführt. Jeder
Gesellschafter ist berechtigt, die Gesellschaft im gewöhnlichen
Geschäftsbetrieb zu vertreten.

\hypertarget{abstimmung}{%
\subsubsection{§ 4.2 -- Abstimmung}\label{abstimmung}}

Sofern in diesem Vertrag nicht anders bestimmt, werden Entscheidungen,
die der Zustimmung der Gesellschafter bedürfen, mit Mehrheit der Stimmen
entsprechend den Gesellschaftsanteilen getroffen. Folgende
Angelegenheiten erfordern Einstimmigkeit:

\begin{enumerate}
\def\labelenumi{(\alph{enumi})}
\tightlist
\item
  Aufnahme neuer Gesellschafter\\
\item
  Änderung dieses Vertrags\\
\item
  Auflösung der Gesellschaft\\
\item
  Verkauf aller oder wesentlicher Teile des Gesellschaftsvermögens\\
\item
  Geschäfte mit Interessenkonflikten
\end{enumerate}

\begin{center}\rule{0.5\linewidth}{0.5pt}\end{center}

\hypertarget{artikel-v-kapitaleinlagen-und-ausschuxfcttungen}{%
\subsection{ARTIKEL V -- KAPITALEINLAGEN UND
AUSSCHÜTTUNGEN}\label{artikel-v-kapitaleinlagen-und-ausschuxfcttungen}}

\hypertarget{anfuxe4ngliche-kapitaleinlagen}{%
\subsubsection{§ 5.1 -- Anfängliche
Kapitaleinlagen}\label{anfuxe4ngliche-kapitaleinlagen}}

Die anfängliche Kapitaleinlage jedes Gesellschafters beträgt:

\begin{quote}
Michael Clement Matejka: \$500,00\\
Ian Bennett Matejka: \$500,00
\end{quote}

\hypertarget{ausschuxfcttungen}{%
\subsubsection{§ 5.2 -- Ausschüttungen}\label{ausschuxfcttungen}}

Ausschüttungen verfügbarer Mittel erfolgen zu den von den
Gesellschaftern bestimmten Zeitpunkten und in der bestimmten Höhe, im
Verhältnis zu den jeweiligen Gesellschaftsanteilen, sofern nicht
einstimmig anders vereinbart.

\hypertarget{thesaurierung}{%
\subsubsection{§ 5.3 -- Thesaurierung}\label{thesaurierung}}

Die Gesellschafter können beschließen, Gewinne für Betriebskapital,
Investitionen oder andere Geschäftszwecke in der Gesellschaft zu
belassen. Solche einbehaltenen Gewinne werden den Kapitalkonten der
Gesellschafter im Verhältnis ihrer Gesellschaftsanteile zugerechnet.

\begin{center}\rule{0.5\linewidth}{0.5pt}\end{center}

\hypertarget{artikel-vi-zukuxfcnftige-erweiterung-der-gesellschafterstruktur}{%
\subsection{ARTIKEL VI -- ZUKÜNFTIGE ERWEITERUNG DER
GESELLSCHAFTERSTRUKTUR}\label{artikel-vi-zukuxfcnftige-erweiterung-der-gesellschafterstruktur}}

\hypertarget{geplante-erweiterung}{%
\subsubsection{§ 6.1 -- Geplante
Erweiterung}\label{geplante-erweiterung}}

Die Gesellschafter bestätigen, dass beabsichtigt ist, künftig weitere
Gesellschafter aufzunehmen, möglicherweise einschließlich
Familienmitglieder bestehender Gesellschafter oder andere qualifizierte
Personen, die zu den Geschäftszielen der Gesellschaft beitragen können.

\hypertarget{verwuxe4sserung-und-anpassung}{%
\subsubsection{§ 6.2 -- Verwässerung und
Anpassung}\label{verwuxe4sserung-und-anpassung}}

Bei Aufnahme neuer Gesellschafter werden die Gesellschaftsanteile der
bestehenden Gesellschafter anteilig angepasst, es sei denn, die
Gesellschafter vereinbaren einstimmig eine andere Aufteilung. Alle
Gesellschafter unterzeichnen eine entsprechende Änderung dieses
Vertrags.

\hypertarget{vorkaufsrecht}{%
\subsubsection{§ 6.3 -- Vorkaufsrecht}\label{vorkaufsrecht}}

Bevor ein Gesellschafter einen Gesellschaftsanteil an einen Dritten
übertragen kann, haben die anderen Gesellschafter ein Vorkaufsrecht zum
Erwerb dieses Anteils zu den gleichen Bedingungen, die der Dritte
anbietet.

\begin{center}\rule{0.5\linewidth}{0.5pt}\end{center}

\hypertarget{artikel-vii-aufluxf6sung-und-abwicklung}{%
\subsection{ARTIKEL VII -- AUFLÖSUNG UND
ABWICKLUNG}\label{artikel-vii-aufluxf6sung-und-abwicklung}}

Die Gesellschaft wird bei Eintritt eines der folgenden Ereignisse
aufgelöst: (a) einstimmige schriftliche Zustimmung aller Gesellschafter;
(b) gerichtliche Auflösungsanordnung; oder (c) jedes andere nach
anwendbarem Recht zur Auflösung führende Ereignis. Nach Auflösung werden
die Angelegenheiten der Gesellschaft gemäß dem Texas Business
Organizations Code abgewickelt.

\begin{center}\rule{0.5\linewidth}{0.5pt}\end{center}

\hypertarget{artikel-viii-schlussbestimmungen}{%
\subsection{ARTIKEL VIII --
SCHLUSSBESTIMMUNGEN}\label{artikel-viii-schlussbestimmungen}}

\hypertarget{anwendbares-recht}{%
\subsubsection{§ 8.1 -- Anwendbares Recht}\label{anwendbares-recht}}

Dieser Vertrag unterliegt dem Recht des Staates Texas unter Ausschluss
der Kollisionsnormen.

\hypertarget{uxe4nderungen}{%
\subsubsection{§ 8.2 -- Änderungen}\label{uxe4nderungen}}

Dieser Vertrag kann nur durch eine von allen Gesellschaftern
unterzeichnete schriftliche Vereinbarung geändert werden.

\hypertarget{vollstuxe4ndigkeit}{%
\subsubsection{§ 8.3 -- Vollständigkeit}\label{vollstuxe4ndigkeit}}

Dieser Vertrag stellt die gesamte Vereinbarung zwischen den
Gesellschaftern hinsichtlich des Vertragsgegenstands dar und ersetzt
alle früheren Vereinbarungen und Absprachen.

\hypertarget{salvatorische-klausel}{%
\subsubsection{§ 8.4 -- Salvatorische
Klausel}\label{salvatorische-klausel}}

Sollte eine Bestimmung dieses Vertrags unwirksam oder undurchsetzbar
sein, bleiben die übrigen Bestimmungen in vollem Umfang wirksam.

\begin{center}\rule{0.5\linewidth}{0.5pt}\end{center}

\hypertarget{unterschriften}{%
\subsection{UNTERSCHRIFTEN}\label{unterschriften}}

\textbf{ZU URKUND DESSEN} haben die unterzeichnenden Gesellschafter
diesen Gesellschaftsvertrag zum oben angegebenen Wirksamkeitsdatum
unterzeichnet.

\textbf{GESELLSCHAFTER:}

\begin{center}\rule{0.5\linewidth}{0.5pt}\end{center}

Michael Clement Matejka\\
Datum: \_\_\_\_\_\_\_\_\_\_\_\_\_\_\_\_\_\_\_\_\_\_\_

\textbf{GESELLSCHAFTER:}

\begin{center}\rule{0.5\linewidth}{0.5pt}\end{center}

Ian Bennett Matejka\\
Datum: \_\_\_\_\_\_\_\_\_\_\_\_\_\_\_\_\_\_\_\_\_\_\_

\begin{center}\rule{0.5\linewidth}{0.5pt}\end{center}

\textbf{BESTÄTIGUNG DES OPTIONSBERECHTIGTEN:}

Der Unterzeichner bestätigt den Erhalt dieses Gesellschaftsvertrags und
erklärt sich bereit, bei Ausübung der Option an die Bestimmungen des
Artikels III (Beteiligungsoption) gebunden zu sein.

\begin{center}\rule{0.5\linewidth}{0.5pt}\end{center}

Wolfram Laube (Optionsberechtigter)\\
Datum: \_\_\_\_\_\_\_\_\_\_\_\_\_\_\_\_\_\_\_\_\_\_\_
