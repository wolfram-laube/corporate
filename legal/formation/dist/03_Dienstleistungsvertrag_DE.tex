\hypertarget{dienstleistungsvertrag}{%
\section{DIENSTLEISTUNGSVERTRAG}\label{dienstleistungsvertrag}}

\emph{{[}Übersetzung -- das englische Original ist rechtlich
maßgeblich{]}}

\begin{center}\rule{0.5\linewidth}{0.5pt}\end{center}

Dieser Dienstleistungsvertrag (der ``\textbf{Vertrag}'') wird zum
\_\_\_\_\_\_\_\_\_\_\_\_\_\_\_\_, 2025 (das
``\textbf{Wirksamkeitsdatum}'') geschlossen zwischen:

\textbf{BLAUWEISS-EDV LLC}, eine Limited Liability Company nach dem
Recht des Staates Texas\\
(nachfolgend ``Auftraggeber'' genannt)

und

\textbf{WOLFRAM LAUBE}\\
Unterer Stadtplatz 18, 4780 Schärding, Österreich\\
(nachfolgend ``Auftragnehmer'' genannt)

(einzeln ``Partei'' und gemeinsam ``Parteien'')

\begin{center}\rule{0.5\linewidth}{0.5pt}\end{center}

\hypertarget{pruxe4ambel}{%
\subsection{PRÄAMBEL}\label{pruxe4ambel}}

Der Auftraggeber ist im Bereich der Erbringung von IT-Dienstleistungen
tätig, einschließlich Softwareentwicklung, Cloud-Architektur,
DevOps-Beratung und Lösungen im Bereich künstlicher Intelligenz.

Der Auftragnehmer verfügt über spezialisierte Fähigkeiten und Expertise
in Softwarearchitektur, Cloud-Infrastruktur, DevOps und KI/Machine
Learning, mit über 25 Jahren Berufserfahrung in der IT-Branche.

Der Auftragnehmer ist derzeit Gegenstand eines Insolvenzverfahrens
(Konkursverfahren GZ 17 S 35/25 s) am Landesgericht Ried im Innkreis,
Österreich, und wünscht eine Erwerbstätigkeit aufzunehmen, die mit den
Anforderungen dieses Verfahrens vereinbar ist.

Die Parteien vereinbaren daher Folgendes:

\begin{center}\rule{0.5\linewidth}{0.5pt}\end{center}

\hypertarget{artikel-1-auftrag-und-leistungen}{%
\subsection{ARTIKEL 1 -- AUFTRAG UND
LEISTUNGEN}\label{artikel-1-auftrag-und-leistungen}}

\hypertarget{beauftragung}{%
\subsubsection{§ 1.1 Beauftragung}\label{beauftragung}}

Der Auftraggeber beauftragt hiermit den Auftragnehmer, und der
Auftragnehmer nimmt diese Beauftragung an, IT-Dienstleistungen für den
Auftraggeber und dessen Kunden zu den in diesem Vertrag festgelegten
Bedingungen zu erbringen.

\hypertarget{leistungsumfang}{%
\subsubsection{§ 1.2 Leistungsumfang}\label{leistungsumfang}}

Der Auftragnehmer erbringt folgende Leistungen (die ``Leistungen''):

\begin{enumerate}
\def\labelenumi{(\alph{enumi})}
\tightlist
\item
  Entwurf und Implementierung von Softwarearchitektur\\
\item
  Cloud-Infrastrukturplanung und -bereitstellung (AWS, Azure, GCP)\\
\item
  DevOps und CI/CD-Pipeline-Entwicklung\\
\item
  Kubernetes-Administration und Container-Orchestrierung\\
\item
  Entwicklung von KI/Machine-Learning-Lösungen\\
\item
  Technische Dokumentation und Beratung\\
\item
  Sonstige einvernehmlich vereinbarte IT-Dienstleistungen
\end{enumerate}

\hypertarget{leistungserbringung}{%
\subsubsection{§ 1.3 Leistungserbringung}\label{leistungserbringung}}

Der Auftragnehmer erbringt die Leistungen remote von Österreich oder
einem anderen vom Auftragnehmer gewählten Standort aus. Der
Auftragnehmer erbringt Leistungen für vom Auftraggeber zugewiesene
Projekte und hält regelmäßigen Kontakt mit dem Auftraggeber bezüglich
Projektstatus und Arbeitsergebnissen.

\begin{center}\rule{0.5\linewidth}{0.5pt}\end{center}

\hypertarget{artikel-2-status-als-selbstuxe4ndiger-auftragnehmer}{%
\subsection{ARTIKEL 2 -- STATUS ALS SELBSTÄNDIGER
AUFTRAGNEHMER}\label{artikel-2-status-als-selbstuxe4ndiger-auftragnehmer}}

\hypertarget{selbstuxe4ndiges-vertragsverhuxe4ltnis}{%
\subsubsection{§ 2.1 Selbständiges
Vertragsverhältnis}\label{selbstuxe4ndiges-vertragsverhuxe4ltnis}}

Der Auftragnehmer ist selbständiger Unternehmer und nicht Arbeitnehmer,
Gesellschafter, Vertreter oder Joint-Venture-Partner des Auftraggebers.
Dieser Vertrag begründet kein Arbeitsverhältnis zwischen den Parteien.
Der Auftragnehmer hat keinen Anspruch auf Arbeitnehmerleistungen des
Auftraggebers.

\hypertarget{keine-vertretungsbefugnis}{%
\subsubsection{§ 2.2 Keine
Vertretungsbefugnis}\label{keine-vertretungsbefugnis}}

Der Auftragnehmer ist nicht befugt, den Auftraggeber zu vertreten oder
Verträge oder Verpflichtungen im Namen des Auftraggebers einzugehen,
sofern nicht ausdrücklich schriftlich ermächtigt.

\hypertarget{steuern-und-sozialabgaben}{%
\subsubsection{§ 2.3 Steuern und
Sozialabgaben}\label{steuern-und-sozialabgaben}}

Der Auftragnehmer ist allein verantwortlich für alle Steuern,
Sozialversicherungsbeiträge und sonstigen gesetzlichen Abgaben, die aus
der Vergütung nach diesem Vertrag entstehen, gemäß den Gesetzen
Österreichs und anderer anwendbarer Rechtsordnungen.

\begin{center}\rule{0.5\linewidth}{0.5pt}\end{center}

\hypertarget{artikel-3-verguxfctung}{%
\subsection{ARTIKEL 3 -- VERGÜTUNG}\label{artikel-3-verguxfctung}}

\hypertarget{verguxfctung-fuxfcr-bescheidene-lebensfuxfchrung}{%
\subsubsection{§ 3.1 Vergütung für bescheidene
Lebensführung}\label{verguxfctung-fuxfcr-bescheidene-lebensfuxfchrung}}

In Anerkennung der aktuellen Umstände des Auftragnehmers, einschließlich
des laufenden Insolvenzverfahrens, vereinbaren die Parteien, dass der
Auftragnehmer eine feste monatliche Vergütung erhält, die ausreicht, um
eine bescheidene Lebensführung zu ermöglichen, wie sie für eine Person
in einem Insolvenzverfahren nach österreichischem Recht angemessen ist.
Der konkrete monatliche Betrag wird in Absprache mit dem
Insolvenzverwalter (APOR Unternehmensverwaltung GmbH) festgelegt und in
einem schriftlichen Nachtrag zu diesem Vertrag dokumentiert.

\hypertarget{zahlungsbedingungen}{%
\subsubsection{§ 3.2 Zahlungsbedingungen}\label{zahlungsbedingungen}}

Die Zahlung erfolgt per Banküberweisung auf ein vom Auftragnehmer
benanntes Konto, zahlbar innerhalb von fünfzehn (15) Tagen nach Ende des
jeweiligen Kalendermonats, in dem Leistungen erbracht wurden.

\hypertarget{wuxe4hrung}{%
\subsubsection{§ 3.3 Währung}\label{wuxe4hrung}}

Alle Vergütungen werden in Euro (EUR) berechnet und gezahlt.

\hypertarget{auslagen}{%
\subsubsection{§ 3.4 Auslagen}\label{auslagen}}

Der Auftraggeber erstattet dem Auftragnehmer angemessene und vorab
genehmigte Auslagen, die im Zusammenhang mit den Leistungen entstehen,
einschließlich Reisekosten, wenn Reisen erforderlich und im Voraus
genehmigt sind.

\hypertarget{abrechnung}{%
\subsubsection{§ 3.5 Abrechnung}\label{abrechnung}}

Der Auftraggeber stellt dem Auftragnehmer monatliche Abrechnungen über
erbrachte Leistungen und gezahlte Vergütungen zur Verfügung. Diese
Abrechnungen sind dem Insolvenzverwalter auf Anfrage zugänglich.

\begin{center}\rule{0.5\linewidth}{0.5pt}\end{center}

\hypertarget{artikel-4-laufzeit-und-kuxfcndigung}{%
\subsection{ARTIKEL 4 -- LAUFZEIT UND
KÜNDIGUNG}\label{artikel-4-laufzeit-und-kuxfcndigung}}

\hypertarget{laufzeit}{%
\subsubsection{§ 4.1 Laufzeit}\label{laufzeit}}

Dieser Vertrag beginnt am Wirksamkeitsdatum und läuft bis zur Kündigung
durch eine der Parteien gemäß diesem Artikel 4.

\hypertarget{ordentliche-kuxfcndigung}{%
\subsubsection{§ 4.2 Ordentliche
Kündigung}\label{ordentliche-kuxfcndigung}}

Jede Partei kann diesen Vertrag jederzeit mit einer Frist von dreißig
(30) Tagen schriftlich gegenüber der anderen Partei kündigen.

\hypertarget{auuxdferordentliche-kuxfcndigung}{%
\subsubsection{§ 4.3 Außerordentliche
Kündigung}\label{auuxdferordentliche-kuxfcndigung}}

Jede Partei kann diesen Vertrag mit sofortiger Wirkung schriftlich
kündigen, wenn die andere Partei wesentliche Bestimmungen dieses
Vertrags verletzt und diese Verletzung nicht innerhalb von fünfzehn (15)
Tagen nach schriftlicher Mitteilung behebt.

\hypertarget{folgen-der-kuxfcndigung}{%
\subsubsection{§ 4.4 Folgen der
Kündigung}\label{folgen-der-kuxfcndigung}}

Bei Kündigung dieses Vertrags: (a) zahlt der Auftraggeber dem
Auftragnehmer alle bis zum Kündigungsdatum verdienten Vergütungen; (b)
übergibt der Auftragnehmer dem Auftraggeber alle Arbeitsergebnisse,
Dokumente und Materialien im Zusammenhang mit den Leistungen; und (c)
gelten die Bestimmungen der Artikel 5, 6 und 7 über die Kündigung hinaus
fort.

\begin{center}\rule{0.5\linewidth}{0.5pt}\end{center}

\hypertarget{artikel-5-vertraulichkeit}{%
\subsection{ARTIKEL 5 --
VERTRAULICHKEIT}\label{artikel-5-vertraulichkeit}}

Der Auftragnehmer verpflichtet sich, alle vertraulichen und proprietären
Informationen des Auftraggebers und dessen Kunden, die im Zusammenhang
mit den Leistungen erlangt werden, vertraulich zu behandeln. Diese
Verpflichtung gilt für einen Zeitraum von fünf (5) Jahren nach
Beendigung dieses Vertrags fort.

\begin{center}\rule{0.5\linewidth}{0.5pt}\end{center}

\hypertarget{artikel-6-geistiges-eigentum}{%
\subsection{ARTIKEL 6 -- GEISTIGES
EIGENTUM}\label{artikel-6-geistiges-eigentum}}

Alle Arbeitsergebnisse, Erfindungen und geistiges Eigentum, die der
Auftragnehmer im Rahmen der Leistungserbringung schafft, sind alleiniges
und ausschließliches Eigentum des Auftraggebers oder des Kunden des
Auftraggebers, je nach Anwendbarkeit. Der Auftragnehmer überträgt
hiermit alle Rechte, Titel und Ansprüche an solchen Arbeitsergebnissen
auf den Auftraggeber.

\begin{center}\rule{0.5\linewidth}{0.5pt}\end{center}

\hypertarget{artikel-7-zusicherungen-und-gewuxe4hrleistungen}{%
\subsection{ARTIKEL 7 -- ZUSICHERUNGEN UND
GEWÄHRLEISTUNGEN}\label{artikel-7-zusicherungen-und-gewuxe4hrleistungen}}

\hypertarget{zusicherungen-des-auftragnehmers}{%
\subsubsection{§ 7.1 Zusicherungen des
Auftragnehmers}\label{zusicherungen-des-auftragnehmers}}

Der Auftragnehmer sichert zu und gewährleistet, dass:

\begin{enumerate}
\def\labelenumi{(\alph{enumi})}
\tightlist
\item
  Der Auftragnehmer dem Auftraggeber das laufende Insolvenzverfahren und
  etwaige Beschränkungen offengelegt hat, die für die Beauftragung des
  Auftragnehmers gelten könnten;\\
\item
  Der Auftragnehmer erforderliche Genehmigungen des Insolvenzverwalters
  vor Aufnahme der Leistungen eingeholt hat oder einholen wird;\\
\item
  Der Auftragnehmer rechtlich befähigt ist, diesen Vertrag abzuschließen
  und die hierin vorgesehenen Leistungen zu erbringen.
\end{enumerate}

\hypertarget{zusicherungen-des-auftraggebers}{%
\subsubsection{§ 7.2 Zusicherungen des
Auftraggebers}\label{zusicherungen-des-auftraggebers}}

Der Auftraggeber sichert zu und gewährleistet, dass er befugt ist,
diesen Vertrag abzuschließen und den Auftragnehmer zu den hierin
festgelegten Bedingungen zu beauftragen.

\begin{center}\rule{0.5\linewidth}{0.5pt}\end{center}

\hypertarget{artikel-8-schlussbestimmungen}{%
\subsection{ARTIKEL 8 --
SCHLUSSBESTIMMUNGEN}\label{artikel-8-schlussbestimmungen}}

\hypertarget{anwendbares-recht}{%
\subsubsection{§ 8.1 Anwendbares Recht}\label{anwendbares-recht}}

Dieser Vertrag unterliegt dem Recht des Staates Texas unter Ausschluss
der Kollisionsnormen.

\hypertarget{uxe4nderungen}{%
\subsubsection{§ 8.2 Änderungen}\label{uxe4nderungen}}

Dieser Vertrag kann nur durch eine von beiden Parteien unterzeichnete
schriftliche Vereinbarung geändert werden.

\hypertarget{vollstuxe4ndigkeit}{%
\subsubsection{§ 8.3 Vollständigkeit}\label{vollstuxe4ndigkeit}}

Dieser Vertrag stellt die gesamte Vereinbarung zwischen den Parteien
hinsichtlich des Vertragsgegenstands dar und ersetzt alle früheren
Vereinbarungen und Absprachen.

\hypertarget{salvatorische-klausel}{%
\subsubsection{§ 8.4 Salvatorische
Klausel}\label{salvatorische-klausel}}

Sollte eine Bestimmung dieses Vertrags unwirksam oder undurchsetzbar
sein, bleiben die übrigen Bestimmungen in vollem Umfang wirksam.

\begin{center}\rule{0.5\linewidth}{0.5pt}\end{center}

\hypertarget{unterschriften}{%
\subsection{UNTERSCHRIFTEN}\label{unterschriften}}

\textbf{ZU URKUND DESSEN} haben die Parteien diesen Vertrag zum oben
angegebenen Wirksamkeitsdatum unterzeichnet.

\textbf{BLAUWEISS-EDV LLC}

Vertreten durch:
\_\_\_\_\_\_\_\_\_\_\_\_\_\_\_\_\_\_\_\_\_\_\_\_\_\_\_\_\_\_\_\_\_\_\_\_\_\_\_\_\_\\
Name: Michael Clement Matejka\\
Position: Geschäftsführender Gesellschafter\\
Datum: \_\_\_\_\_\_\_\_\_\_\_\_\_\_\_\_\_\_\_\_\_\_\_

\textbf{AUFTRAGNEHMER}

\begin{center}\rule{0.5\linewidth}{0.5pt}\end{center}

Wolfram Laube\\
Datum: \_\_\_\_\_\_\_\_\_\_\_\_\_\_\_\_\_\_\_\_\_\_\_
