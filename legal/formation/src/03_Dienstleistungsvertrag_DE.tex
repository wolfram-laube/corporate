\documentclass[11pt,a4paper]{article}
\usepackage[ngerman]{babel}
\usepackage[utf8]{inputenc}
\usepackage[T1]{fontenc}
\usepackage{geometry}
\usepackage{parskip}
\usepackage{enumitem}
\usepackage{titlesec}

\geometry{margin=2.5cm}
\titleformat{\section}{\large\bfseries}{\thesection}{1em}{}
\titleformat{\subsection}{\normalsize\bfseries}{\thesubsection}{1em}{}

\begin{document}

\begin{center}
{\Large\bfseries DIENSTLEISTUNGSVERTRAG}\\[0.5em]
{\small\textit{[Übersetzung -- das englische Original ist rechtlich maßgeblich]}}
\end{center}

\vspace{1em}

Dieser Dienstleistungsvertrag (der „\textbf{Vertrag}") wird zum \underline{\hspace{3cm}}, 2025 (das „\textbf{Wirksamkeitsdatum}") geschlossen zwischen:

\textbf{BLAUWEISS-EDV LLC}, eine Limited Liability Company nach dem Recht des Staates Texas\\
(nachfolgend „Auftraggeber" genannt)

und

\textbf{WOLFRAM LAUBE}\\
Unterer Stadtplatz 18, 4780 Schärding, Österreich\\
(nachfolgend „Auftragnehmer" genannt)

(einzeln „Partei" und gemeinsam „Parteien")

\section*{PRÄAMBEL}

Der Auftraggeber ist im Bereich der Erbringung von IT-Dienstleistungen tätig, einschließlich Softwareentwicklung, Cloud-Architektur, DevOps-Beratung und Lösungen im Bereich künstlicher Intelligenz.

Der Auftragnehmer verfügt über spezialisierte Fähigkeiten und Expertise in Softwarearchitektur, Cloud-Infrastruktur, DevOps und KI/Machine Learning, mit über 25 Jahren Berufserfahrung in der IT-Branche.

Der Auftragnehmer ist derzeit Gegenstand eines Insolvenzverfahrens (Konkursverfahren GZ 17 S 35/25 s) am Landesgericht Ried im Innkreis, Österreich, und wünscht eine Erwerbstätigkeit aufzunehmen, die mit den Anforderungen dieses Verfahrens vereinbar ist.

Die Parteien vereinbaren daher Folgendes:

\section{AUFTRAG UND LEISTUNGEN}

\subsection{Beauftragung}
Der Auftraggeber beauftragt hiermit den Auftragnehmer, und der Auftragnehmer nimmt diese Beauftragung an, IT-Dienstleistungen für den Auftraggeber und dessen Kunden zu den in diesem Vertrag festgelegten Bedingungen zu erbringen.

\subsection{Leistungsumfang}
Der Auftragnehmer erbringt folgende Leistungen (die „Leistungen"): (a) Entwurf und Implementierung von Softwarearchitektur; (b) Cloud-Infrastrukturplanung und -bereitstellung (AWS, Azure, GCP); (c) DevOps und CI/CD-Pipeline-Entwicklung; (d) Kubernetes-Administration und Container-Orchestrierung; (e) Entwicklung von KI/Machine-Learning-Lösungen; (f) Technische Dokumentation und Beratung; (g) Sonstige einvernehmlich vereinbarte IT-Dienstleistungen.

\subsection{Leistungserbringung}
Der Auftragnehmer erbringt die Leistungen remote von Österreich oder einem anderen vom Auftragnehmer gewählten Standort aus.

\section{STATUS ALS SELBSTÄNDIGER AUFTRAGNEHMER}

\subsection{Selbständiges Vertragsverhältnis}
Der Auftragnehmer ist selbständiger Unternehmer und nicht Arbeitnehmer, Gesellschafter, Vertreter oder Joint-Venture-Partner des Auftraggebers. Dieser Vertrag begründet kein Arbeitsverhältnis zwischen den Parteien.

\subsection{Keine Vertretungsbefugnis}
Der Auftragnehmer ist nicht befugt, den Auftraggeber zu vertreten oder Verträge im Namen des Auftraggebers einzugehen, sofern nicht ausdrücklich schriftlich ermächtigt.

\subsection{Steuern und Sozialabgaben}
Der Auftragnehmer ist allein verantwortlich für alle Steuern, Sozialversicherungsbeiträge und sonstigen gesetzlichen Abgaben, die aus der Vergütung nach diesem Vertrag entstehen.

\section{VERGÜTUNG}

\subsection{Vergütung für bescheidene Lebensführung}
In Anerkennung der aktuellen Umstände des Auftragnehmers, einschließlich des laufenden Insolvenzverfahrens, vereinbaren die Parteien, dass der Auftragnehmer eine feste monatliche Vergütung erhält, die ausreicht, um eine bescheidene Lebensführung zu ermöglichen, wie sie für eine Person in einem Insolvenzverfahren nach österreichischem Recht angemessen ist. Der konkrete monatliche Betrag wird in Absprache mit dem Insolvenzverwalter (APOR Unternehmensverwaltung GmbH) festgelegt und in einem schriftlichen Nachtrag zu diesem Vertrag dokumentiert.

\subsection{Zahlungsbedingungen}
Die Zahlung erfolgt per Banküberweisung auf ein vom Auftragnehmer benanntes Konto, zahlbar innerhalb von fünfzehn (15) Tagen nach Ende des jeweiligen Kalendermonats, in dem Leistungen erbracht wurden.

\subsection{Währung}
Alle Vergütungen werden in Euro (EUR) berechnet und gezahlt.

\subsection{Auslagen}
Der Auftraggeber erstattet dem Auftragnehmer angemessene und vorab genehmigte Auslagen, die im Zusammenhang mit den Leistungen entstehen.

\subsection{Abrechnung}
Der Auftraggeber stellt dem Auftragnehmer monatliche Abrechnungen über erbrachte Leistungen und gezahlte Vergütungen zur Verfügung. Diese Abrechnungen sind dem Insolvenzverwalter auf Anfrage zugänglich.

\section{LAUFZEIT UND KÜNDIGUNG}

\subsection{Laufzeit}
Dieser Vertrag beginnt am Wirksamkeitsdatum und läuft bis zur Kündigung durch eine der Parteien gemäß diesem Artikel.

\subsection{Ordentliche Kündigung}
Jede Partei kann diesen Vertrag jederzeit mit einer Frist von dreißig (30) Tagen schriftlich gegenüber der anderen Partei kündigen.

\subsection{Außerordentliche Kündigung}
Jede Partei kann diesen Vertrag mit sofortiger Wirkung schriftlich kündigen, wenn die andere Partei wesentliche Bestimmungen dieses Vertrags verletzt und diese Verletzung nicht innerhalb von fünfzehn (15) Tagen nach schriftlicher Mitteilung behebt.

\subsection{Folgen der Kündigung}
Bei Kündigung dieses Vertrags: (a) zahlt der Auftraggeber dem Auftragnehmer alle bis zum Kündigungsdatum verdienten Vergütungen; (b) übergibt der Auftragnehmer dem Auftraggeber alle Arbeitsergebnisse und Materialien; und (c) gelten die Bestimmungen der Artikel 5, 6 und 7 über die Kündigung hinaus fort.

\section{VERTRAULICHKEIT}

Der Auftragnehmer verpflichtet sich, alle vertraulichen und proprietären Informationen des Auftraggebers und dessen Kunden vertraulich zu behandeln. Diese Verpflichtung gilt für fünf (5) Jahre nach Beendigung dieses Vertrags fort.

\section{GEISTIGES EIGENTUM}

Alle Arbeitsergebnisse, Erfindungen und geistiges Eigentum, die der Auftragnehmer im Rahmen der Leistungserbringung schafft, sind alleiniges Eigentum des Auftraggebers. Der Auftragnehmer überträgt hiermit alle Rechte an solchen Arbeitsergebnissen auf den Auftraggeber.

\section{ZUSICHERUNGEN UND GEWÄHRLEISTUNGEN}

\subsection{Zusicherungen des Auftragnehmers}
Der Auftragnehmer sichert zu und gewährleistet, dass: (a) Der Auftragnehmer das laufende Insolvenzverfahren offengelegt hat; (b) Der Auftragnehmer erforderliche Genehmigungen des Insolvenzverwalters einholen wird; (c) Der Auftragnehmer rechtlich befähigt ist, diesen Vertrag abzuschließen.

\subsection{Zusicherungen des Auftraggebers}
Der Auftraggeber sichert zu und gewährleistet, dass er befugt ist, diesen Vertrag abzuschließen.

\section{SCHLUSSBESTIMMUNGEN}

\subsection{Anwendbares Recht}
Dieser Vertrag unterliegt dem Recht des Staates Texas.

\subsection{Änderungen}
Dieser Vertrag kann nur durch eine von beiden Parteien unterzeichnete schriftliche Vereinbarung geändert werden.

\subsection{Vollständigkeit}
Dieser Vertrag stellt die gesamte Vereinbarung zwischen den Parteien dar.

\subsection{Salvatorische Klausel}
Sollte eine Bestimmung dieses Vertrags unwirksam sein, bleiben die übrigen Bestimmungen in vollem Umfang wirksam.

\vspace{2em}
\section*{UNTERSCHRIFTEN}

\textbf{ZU URKUND DESSEN} haben die Parteien diesen Vertrag zum oben angegebenen Wirksamkeitsdatum unterzeichnet.

\vspace{2em}
\textbf{BLAUWEISS-EDV LLC}

\vspace{1em}
Vertreten durch: \underline{\hspace{6cm}}\\
Name: Michael Clement Matejka\\
Position: Geschäftsführender Gesellschafter\\
Datum: \underline{\hspace{4cm}}

\vspace{2em}
\textbf{AUFTRAGNEHMER}

\vspace{1em}
\underline{\hspace{6cm}}\\
Wolfram Laube\\
Datum: \underline{\hspace{4cm}}

\end{document}
