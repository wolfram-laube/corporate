\documentclass[11pt,a4paper]{article}
\usepackage[utf8]{inputenc}
\usepackage[T1]{fontenc}
\usepackage[ngerman]{babel}
\usepackage{geometry}
\usepackage{parskip}
\usepackage{enumitem}
\usepackage{titlesec}

\geometry{margin=2.5cm}

\titleformat{\section}{\large\bfseries}{ARTIKEL \Roman{section} -- }{0em}{}
\titleformat{\subsection}{\normalsize\bfseries}{§ \thesection.\arabic{subsection} -- }{0em}{}

\begin{document}

\textit{[Übersetzung zu Dokumentationszwecken -- das englische Original ist rechtlich maßgeblich]}

\vspace{0.5cm}

\begin{center}
{\Large\textbf{GESELLSCHAFTSVERTRAG}}\\[0.2cm]
(Operating Agreement)\\[0.3cm]
der\\[0.3cm]
{\large\textbf{BLAUWEISS-EDV LLC}}\\[0.3cm]
Eine Limited Liability Company nach dem Recht des Staates Texas
\end{center}

\vspace{1cm}

Dieser Gesellschaftsvertrag (das ``\textbf{Agreement}'') wird zum \underline{\hspace{3cm}}, 2025 (das ``\textbf{Wirksamkeitsdatum}'') zwischen den hierin bezeichneten Gesellschaftern geschlossen.

\textbf{PRÄAMBEL:} Die Blauweiss-EDV LLC (die ``\textbf{Gesellschaft}'') wurde am 6. November 2025 als Limited Liability Company nach dem Recht des Staates Texas durch Einreichung einer Gründungsurkunde beim Texas Secretary of State gegründet (Aktenzeichen 806293404).

Die Gesellschafter beabsichtigen, die Bedingungen für den Betrieb der Gesellschaft sowie die Rechte und Pflichten der Gesellschafter festzulegen.

\section{ORGANISATION}

\subsection{Firma}
Die Firma der Gesellschaft lautet Blauweiss-EDV LLC.

\subsection{Hauptsitz}
Der Hauptsitz der Gesellschaft befindet sich an einem von den Gesellschaftern jeweils zu bestimmenden Ort. Das eingetragene Büro entspricht den Angaben in der Gründungsurkunde.

\subsection{Unternehmensgegenstand}
Gegenstand der Gesellschaft ist die Erbringung von IT-Dienstleistungen, einschließlich aber nicht beschränkt auf Softwareentwicklung, Cloud-Architektur, DevOps-Beratung, Lösungen im Bereich künstlicher Intelligenz sowie alle anderen von den Gesellschaftern bestimmten rechtmäßigen Geschäftstätigkeiten.

\subsection{Dauer}
Die Gesellschaft besteht auf unbestimmte Zeit, sofern sie nicht gemäß diesem Vertrag oder anwendbarem Recht aufgelöst wird.

\section{GESELLSCHAFTER UND GESELLSCHAFTSANTEILE}

\subsection{Gründungsgesellschafter}
Die Gründungsgesellschafter der Gesellschaft und ihre jeweiligen Gesellschaftsanteile sind wie folgt:

\begin{quote}
\textbf{Michael Clement Matejka} -- 50\% Gesellschaftsanteil\\
106 Stratford St, Houston, TX 77006, USA\\
US-Staatsbürger

\textbf{Ian Bennett Matejka} -- 50\% Gesellschaftsanteil\\
Houston, Texas, USA\\
US-Staatsbürger
\end{quote}

\subsection{Aufnahme weiterer Gesellschafter}
Weitere Gesellschafter können mit einstimmiger schriftlicher Zustimmung aller bestehenden Gesellschafter und durch Unterzeichnung einer Änderung dieses Vertrags aufgenommen werden, die die Aufnahme des neuen Gesellschafters und die Anpassung der Gesellschaftsanteile widerspiegelt.

\section{BETEILIGUNGSOPTION}

\subsection{Einräumung der Option}
Die Gesellschafter räumen hiermit \textbf{Wolfram Laube} (``\textbf{Optionsberechtigter}''), einem österreichischen Staatsbürger wohnhaft in Unterer Stadtplatz 18, 4780 Schärding, Österreich, eine Option (die ``\textbf{Beteiligungsoption}'') zum Erwerb eines Gesellschaftsanteils an der Gesellschaft ein, vorbehaltlich der in diesem Artikel III festgelegten Bedingungen.

\subsection{Ausübungsbedingungen}
Die Beteiligungsoption kann nur bei Erfüllung der folgenden Bedingung ausgeübt werden: Der Abschluss des derzeit gegen den Optionsberechtigten beim Landesgericht Ried im Innkreis, Österreich, anhängigen Insolvenzverfahrens (Konkursverfahren GZ 17 S 35/25 s) durch rechtskräftigen Gerichtsbeschluss, der die Beendigung oder Aufhebung dieses Verfahrens bestätigt.

\subsection{Ausübungsverfahren}
Nach Erfüllung der Ausübungsbedingung kann der Optionsberechtigte die Beteiligungsoption durch schriftliche Mitteilung an alle Gesellschafter ausüben, begleitet von Dokumenten, die den Abschluss des Insolvenzverfahrens belegen. Bei gültiger Ausübung wird der Optionsberechtigte als Gesellschafter aufgenommen, und die Gesellschaftsanteile werden so angepasst, dass jeder der drei Gesellschafter einen Anteil von 33,33\% hält (oder eine andere von allen Parteien einstimmig vereinbarte Aufteilung).

\subsection{Ausübungspreis}
Der Ausübungspreis für die Beteiligungsoption beträgt einen US-Dollar (\$1,00) als Gegenleistung für die Beiträge des Optionsberechtigten zur Gesellschaft als Independent Contractor und in Anerkennung seiner Rolle beim Aufbau der Geschäftsbeziehungen und technischen Fähigkeiten der Gesellschaft.

\subsection{Optionsfrist}
Die Beteiligungsoption bleibt für einen Zeitraum von zwei (2) Jahren nach Erfüllung der Ausübungsbedingung ausübbar. Wird sie innerhalb dieses Zeitraums nicht ausgeübt, verfällt die Beteiligungsoption.

\subsection{Contractor-Verhältnis}
Bis zur Ausübung der Beteiligungsoption erbringt der Optionsberechtigte Leistungen für die Gesellschaft als Independent Contractor gemäß einem separaten Independent Contractor Agreement. Dieses Vertragsverhältnis begründet keine Gesellschafterrechte oder -pflichten vor gültiger Ausübung der Beteiligungsoption.

\section{GESCHÄFTSFÜHRUNG}

\subsection{Gesellschaftergeführte Gesellschaft}
Die Gesellschaft wird von ihren Gesellschaftern geführt. Jeder Gesellschafter ist berechtigt, die Gesellschaft im gewöhnlichen Geschäftsbetrieb zu vertreten.

\subsection{Abstimmung}
Sofern in diesem Vertrag nicht anders bestimmt, werden Entscheidungen, die der Zustimmung der Gesellschafter bedürfen, mit Mehrheit der Stimmen entsprechend den Gesellschaftsanteilen getroffen. Folgende Angelegenheiten erfordern Einstimmigkeit:

\begin{enumerate}[label=(\alph*)]
    \item Aufnahme neuer Gesellschafter
    \item Änderung dieses Vertrags
    \item Auflösung der Gesellschaft
    \item Verkauf aller oder wesentlicher Teile des Gesellschaftsvermögens
    \item Geschäfte mit Interessenkonflikten
\end{enumerate}

\section{KAPITALEINLAGEN UND AUSSCHÜTTUNGEN}

\subsection{Anfängliche Kapitaleinlagen}
Die anfängliche Kapitaleinlage jedes Gesellschafters beträgt:

\begin{quote}
Michael Clement Matejka: \$500,00\\
Ian Bennett Matejka: \$500,00
\end{quote}

\subsection{Ausschüttungen}
Ausschüttungen verfügbarer Mittel erfolgen zu den von den Gesellschaftern bestimmten Zeitpunkten und in der bestimmten Höhe, im Verhältnis zu den jeweiligen Gesellschaftsanteilen, sofern nicht einstimmig anders vereinbart.

\subsection{Thesaurierung}
Die Gesellschafter können beschließen, Gewinne für Betriebskapital, Investitionen oder andere Geschäftszwecke in der Gesellschaft zu belassen. Solche einbehaltenen Gewinne werden den Kapitalkonten der Gesellschafter im Verhältnis ihrer Gesellschaftsanteile zugerechnet.

\section{ZUKÜNFTIGE ERWEITERUNG DER GESELLSCHAFTERSTRUKTUR}

\subsection{Geplante Erweiterung}
Die Gesellschafter bestätigen, dass beabsichtigt ist, künftig weitere Gesellschafter aufzunehmen, möglicherweise einschließlich Familienmitglieder bestehender Gesellschafter oder andere qualifizierte Personen, die zu den Geschäftszielen der Gesellschaft beitragen können.

\subsection{Verwässerung und Anpassung}
Bei Aufnahme neuer Gesellschafter werden die Gesellschaftsanteile der bestehenden Gesellschafter anteilig angepasst, es sei denn, die Gesellschafter vereinbaren einstimmig eine andere Aufteilung. Alle Gesellschafter unterzeichnen eine entsprechende Änderung dieses Vertrags.

\subsection{Vorkaufsrecht}
Bevor ein Gesellschafter einen Gesellschaftsanteil an einen Dritten übertragen kann, haben die anderen Gesellschafter ein Vorkaufsrecht zum Erwerb dieses Anteils zu den gleichen Bedingungen, die der Dritte anbietet.

\section{AUFLÖSUNG UND ABWICKLUNG}

Die Gesellschaft wird bei Eintritt eines der folgenden Ereignisse aufgelöst: (a) einstimmige schriftliche Zustimmung aller Gesellschafter; (b) gerichtliche Auflösungsanordnung; oder (c) jedes andere nach anwendbarem Recht zur Auflösung führende Ereignis. Nach Auflösung werden die Angelegenheiten der Gesellschaft gemäß dem Texas Business Organizations Code abgewickelt.

\section{SCHLUSSBESTIMMUNGEN}

\subsection{Anwendbares Recht}
Dieser Vertrag unterliegt dem Recht des Staates Texas unter Ausschluss der Kollisionsnormen.

\subsection{Änderungen}
Dieser Vertrag kann nur durch eine von allen Gesellschaftern unterzeichnete schriftliche Vereinbarung geändert werden.

\subsection{Vollständigkeit}
Dieser Vertrag stellt die gesamte Vereinbarung zwischen den Gesellschaftern hinsichtlich des Vertragsgegenstands dar und ersetzt alle früheren Vereinbarungen und Absprachen.

\subsection{Salvatorische Klausel}
Sollte eine Bestimmung dieses Vertrags unwirksam oder undurchsetzbar sein, bleiben die übrigen Bestimmungen in vollem Umfang wirksam.

\vspace{1cm}

\textbf{ZU URKUND DESSEN} haben die unterzeichnenden Gesellschafter diesen Gesellschaftsvertrag zum oben angegebenen Wirksamkeitsdatum unterzeichnet.

\vspace{1.5cm}

\textbf{GESELLSCHAFTER:}

\vspace{1.5cm}

\underline{\hspace{8cm}}\\
Michael Clement Matejka\\
Datum: \underline{\hspace{4cm}}

\vspace{1.5cm}

\textbf{GESELLSCHAFTER:}

\vspace{1.5cm}

\underline{\hspace{8cm}}\\
Ian Bennett Matejka\\
Datum: \underline{\hspace{4cm}}

\vspace{1.5cm}

\textbf{BESTÄTIGUNG DES OPTIONSBERECHTIGTEN:}

Der Unterzeichner bestätigt den Erhalt dieses Gesellschaftsvertrags und erklärt sich bereit, bei Ausübung der Option an die Bestimmungen des Artikels III (Beteiligungsoption) gebunden zu sein.

\vspace{1.5cm}

\underline{\hspace{8cm}}\\
Wolfram Laube (Optionsberechtigter)\\
Datum: \underline{\hspace{4cm}}

\end{document}
